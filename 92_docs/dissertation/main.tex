% !TeX program = pdflatex
\documentclass[12pt]{article}
\usepackage[a4paper,margin=1in]{geometry}
\usepackage[T1]{fontenc}
\usepackage{lmodern}
\usepackage{amsmath,amssymb,amsthm}
\usepackage{physics}
\usepackage{graphicx}
\usepackage{hyperref}
\usepackage{csquotes}
\usepackage[backend=biber,style=alphabetic]{biblatex}
\addbibresource{references.bib}

\title{Amon--Turing Field Theory: Foundations, Derivations, and Implications}
\author{Christopher Eklectic}
\date{\today}

\begin{document}
\maketitle
\begin{abstract}
We present a unified field framework on a primordial substrate using five core operators to derive classical mechanics, quantum mechanics, and aspects of biological organization. This dissertation formalizes operator definitions with dimensional consistency, develops derivations to known laws, proposes experimental tests, and outlines instrumentation via the Coherence Modal Cross-Interferometer (CMCI).
\end{abstract}

\tableofcontents
\listoffigures
\listoftables

\section{Introduction}
Motivation, scope, and contributions. Outline of the dissertation structure.

\subsection{Thesis and Claims}
This work proposes that a primitive substrate $\Phi_0$ supporting a field $\Psi$ and a minimal set of operators (Point $\hat P$, Line $\hat L$, Curve $\hat C$, Movement $\hat M$, Resistance $\hat R$) yields:
\begin{itemize}
  \item A conservative baseline that reduces to a scalar wave equation with a well-defined variational principle and Noether currents.
  \item Recoveries of classical and quantum dynamics under appropriate limits and ans"atze (Newtonian limit; uncertainty via Cauchy--Schwarz; mapping to Schr"odinger-like evolution under slowly varying envelope approximations).
  \item Coherence-centric predictions linking spatial structure to dynamical stability, testable via a Coherence Modal Cross-Interferometer (CMCI).
  \item A reproducible modeling and measurement pipeline producing falsifiable, quantitative predictions.
\end{itemize}

\section{Background}
Related work in unification attempts, morphogenetic fields, and coherence phenomena.

\section{Mathematical Framework}
\subsection{Substrate and Fields}
Define substrate $\Phi_0$ and field $\Psi$; domain, boundary conditions, normalization conventions.
\subsection{Operators}
Point ($\hat P$), Line ($\hat L$), Curve ($\hat C$), Movement ($\hat M$), Resistance ($\hat R$) with units and commutation relations.

\subsection{Variational Formulation}
We introduce a conservative baseline in which dissipation is absent ($\hat R=0$) and the dynamics follow from a Lagrangian density
\begin{equation}
  \mathcal{L}[\Psi] = \tfrac{1}{2}\, (\partial_t \Psi)^2 - \tfrac{c^2}{2}\, \lVert\nabla \Psi\rVert^2 + S\,\Psi\, ,
\end{equation}
with source $S$ and characteristic propagation speed $c$. The Euler--Lagrange equation yields
\begin{equation}
  \partial_t^2 \Psi - c^2\,\nabla^2\Psi = S\, ,
\end{equation}
which we identify with the Curve (spatial Laplacian) and Movement (time-derivative) operators. Conservation of energy and momentum follows from time and space translation symmetries (Noether's theorem). Phenomenological resistance is incorporated via a Rayleigh dissipation function $\mathcal{R} = \tfrac{k}{2}\,(\partial_t\Psi)^2$ leading to an additional term $-k\,\partial_t\Psi$ in the equation of motion.

\section{Derivations}
\subsection{Classical Mechanics}
Newton's second law from the limiting behavior of the master equation.
\subsection{Quantum Mechanics}
Uncertainty relation via Cauchy--Schwarz; mapping to Schr"odinger dynamics.
\subsection{Mass--Energy}
Rest energy from stationary states; emergence of $c$ from substrate propagation.

\section{Biological Organization}
Self-referential loops, coherence metrics, and information flow.

\section{Experimental Proposals}
Quantum biology signatures, neural coherence, and gravitational anomalies.

\section{Instrumentation: CMCI}
Architecture, acquisition chain, and measurement protocols for cross-spectrum coherence.

\section{Simulation Methods}
Numerical reference implementation uses a two-dimensional grid and an explicit scheme for the damped wave equation
\begin{equation}
  \partial_t^2 \Psi = c^2\,\nabla^2\Psi - k\,\partial_t\Psi + S\, .
\end{equation}
Convergence and stability are documented; energy conservation is verified in the $k{=}0$ limit. See repository path `20_modeling/simulations/` for code and configurations.

\section{Discussion}
Limitations, falsifiable predictions, and future work.

\printbibliography

\end{document}
