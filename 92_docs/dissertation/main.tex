% !TeX program = pdflatex
\documentclass[12pt]{article}
\usepackage[a4paper,margin=1in]{geometry}
\usepackage[T1]{fontenc}
\usepackage{lmodern}
\usepackage{amsmath,amssymb,amsthm}
\usepackage{physics}
\usepackage{graphicx}
\usepackage{hyperref}
\usepackage{csquotes}
\usepackage[backend=biber,style=alphabetic]{biblatex}
\addbibresource{references.bib}

\title{Amon--Turing Field Theory: Foundations, Derivations, and Implications}
\author{Christopher Eklectic}
\date{\today}

\begin{document}
\maketitle
\begin{abstract}
We present a unified field framework on a primordial substrate using five core operators to derive classical mechanics, quantum mechanics, and aspects of biological organization. This dissertation formalizes operator definitions with dimensional consistency, develops derivations to known laws, proposes experimental tests, and outlines instrumentation via the Coherence Modal Cross-Interferometer (CMCI).
\end{abstract}

\tableofcontents
\listoffigures
\listoftables

\section{Introduction}
Motivation, scope, and contributions. Outline of the dissertation structure.

\section{Background}
Related work in unification attempts, morphogenetic fields, and coherence phenomena.

\section{Mathematical Framework}
\subsection{Substrate and Fields}
Define substrate $\Phi_0$ and field $\Psi$; domain, boundary conditions, normalization conventions.
\subsection{Operators}
Point ($\hat P$), Line ($\hat L$), Curve ($\hat C$), Movement ($\hat M$), Resistance ($\hat R$) with units and commutation relations.

\section{Derivations}
\subsection{Classical Mechanics}
Newton's second law from the limiting behavior of the master equation.
\subsection{Quantum Mechanics}
Uncertainty relation via Cauchy--Schwarz; mapping to Schr"odinger dynamics.
\subsection{Mass--Energy}
Rest energy from stationary states; emergence of $c$ from substrate propagation.

\section{Biological Organization}
Self-referential loops, coherence metrics, and information flow.

\section{Experimental Proposals}
Quantum biology signatures, neural coherence, and gravitational anomalies.

\section{Instrumentation: CMCI}
Architecture, acquisition chain, and measurement protocols for cross-spectrum coherence.

\section{Discussion}
Limitations, falsifiable predictions, and future work.

\printbibliography

\end{document}

