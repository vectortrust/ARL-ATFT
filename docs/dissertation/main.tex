% !TeX program = pdflatex
\documentclass[12pt]{article}
\usepackage[a4paper,margin=1in]{geometry}
\usepackage[T1]{fontenc}
\usepackage{lmodern}
\usepackage{amsmath,amssymb,amsthm}
\usepackage{physics}
\usepackage{graphicx}
\usepackage{hyperref}
\usepackage{csquotes}
\usepackage[backend=biber,style=alphabetic]{biblatex}
\addbibresource{references.bib}

\title{Amon--Turing Field Theory: Foundations, Derivations, and Implications}
\author{Christopher Eklectic}
\date{\today}

\begin{document}
\maketitle
\begin{abstract}
We present a unified field framework on a primordial substrate using five core operators to derive classical mechanics, quantum mechanics, and aspects of biological organization. This dissertation formalizes operator definitions with dimensional consistency, develops derivations to known laws, proposes experimental tests, and outlines instrumentation via the Coherence Modal Cross-Interferometer (CMCI).
\end{abstract}

\tableofcontents
\listoffigures
\listoftables

\section{Introduction}
Motivation, scope, and contributions. Outline of the dissertation structure.

\subsection{Thesis and Claims}
This work proposes that a primitive substrate $\Phi_0$ supporting a field $\Psi$ and a minimal set of operators (Point $\hat P$, Line $\hat L$, Curve $\hat C$, Movement $\hat M$, Resistance $\hat R$) yields:
\begin{itemize}
  \item A conservative baseline that reduces to a scalar wave equation with a well-defined variational principle and Noether currents.
  \item Recoveries of classical and quantum dynamics under appropriate limits and ans"atze (Newtonian limit; uncertainty via Cauchy--Schwarz; mapping to Schr"odinger-like evolution under slowly varying envelope approximations).
  \item Coherence-centric predictions linking spatial structure to dynamical stability, testable via a Coherence Modal Cross-Interferometer (CMCI).
  \item A reproducible modeling and measurement pipeline producing falsifiable, quantitative predictions.
\end{itemize}

\section{Background}
Related work in unification attempts, morphogenetic fields, and coherence phenomena.

\section{Mathematical Framework}
\subsection{Substrate and Fields}
Define substrate $\Phi_0$ and field $\Psi$; domain, boundary conditions, normalization conventions.
\subsection{Operators}
Point ($\hat P$), Line ($\hat L$), Curve ($\hat C$), Movement ($\hat M$), Resistance ($\hat R$) with units and commutation relations.

\paragraph{Dimensions and Units.}
Let $[L]$ and $[T]$ denote the base dimensions of length and time. Let the field carry units $[\Psi]=[A]$ (an abstract amplitude).
We define the operator actions and their unit effects on $\Psi$:
\begin{itemize}
  \item Movement $\hat M\equiv \partial_t$ so that $[\hat M\,\Psi]=[A] [T]^{-1}$.
  \item Curve $\hat C\equiv \nabla^2$ so that $[\hat C\,\Psi]=[A] [L]^{-2}$.
  \item Line $\hat L\equiv \nabla\cdot (\cdot)$ acting on vectorial constructions from $\Psi$; baseline $[\hat L\,\Psi]=[A] [L]^{-1}$.
  \item Point $\hat P$ selects or injects localized contributions; in the continuum, modeled by sources $S$ with $[S]=[A][T]^{-2}$ to match the equation of motion.
  \item Resistance introduces damping $-k\,\partial_t\Psi$ with $[k]=[T]^{-1}$.
\end{itemize}

\paragraph{Commutators and Domains.}
On sufficiently smooth functions with constant coefficients, spatial and temporal derivatives commute: $[\partial_t,\nabla]=0$, hence $[\hat M,\hat C]=0$ and $[\hat M,\hat L]=0$ in the baseline. Position- or time-dependent coefficients (e.g., $c=c(x,t)$ or $k=k(x,t)$) induce nonzero commutators and additional terms under variation. Function spaces (Sobolev spaces $H^s$ on bounded domains with boundary conditions) provide domains where $\hat C$ is self-adjoint (e.g., with Dirichlet/Neumann BCs), enabling spectral analysis.

\subsection{Variational Formulation}
We introduce a conservative baseline in which dissipation is absent ($\hat R=0$) and the dynamics follow from a Lagrangian density
\begin{equation}
  \mathcal{L}[\Psi] = \tfrac{1}{2}\, (\partial_t \Psi)^2 - \tfrac{c^2}{2}\, \lVert\nabla \Psi\rVert^2 + S\,\Psi\, ,
\end{equation}
with source $S$ and characteristic propagation speed $c$. The Euler--Lagrange equation yields
\begin{equation}
  \partial_t^2 \Psi - c^2\,\nabla^2\Psi = S\, ,
\end{equation}
which we identify with the Curve (spatial Laplacian) and Movement (time-derivative) operators. Conservation of energy and momentum follows from time and space translation symmetries (Noether's theorem). Phenomenological resistance is incorporated via a Rayleigh dissipation function $\mathcal{R} = \tfrac{k}{2}\,(\partial_t\Psi)^2$ leading to an additional term $-k\,\partial_t\Psi$ in the equation of motion.

\section{Derivations}
\subsection{Classical Mechanics}
\textbf{Newtonian limit.} Consider localized wave packets with a slowly varying envelope. Under a multi-scale expansion, the wave equation admits an effective particle trajectory whose centroid obeys $\ddot x = -\nabla V/m$ when sources encode a potential $V$ and $c$ is large compared to envelope dynamics. We provide explicit steps and small-parameter ordering in the appendix and verify numerically with wave-packet propagation tests.
\subsection{Quantum Mechanics}
\textbf{Uncertainty.} For square-integrable $\Psi$ with appropriate normalization, Cauchy--Schwarz yields $\Delta x\, \Delta p \ge \tfrac{1}{2}$ in rescaled units; restoring constants gives the familiar $\Delta x\,\Delta p\ge \hbar/2$ in the Schr"odinger mapping.

\textbf{Schr"odinger mapping.} Introduce a complex envelope $\psi$ via a standard ansatz $\Psi(x,t)=\Re\{ \psi(x,t)\, e^{-i\omega_0 t} \}$ and perform a slowly varying envelope approximation (SVEA). Keeping leading-order terms produces a parabolic equation of the Schr"odinger type for $\psi$ with an effective mass and potential inherited from source terms; details and validity conditions are given in the appendix.
\subsection{Mass--Energy}
Rest energy from stationary states; emergence of $c$ from substrate propagation.

\section{Biological Organization}
Self-referential loops, coherence metrics, and information flow.

\section{Experimental Proposals}
Quantum biology signatures, neural coherence, and gravitational anomalies.

\section{Instrumentation: CMCI}
Architecture, acquisition chain, and measurement protocols for cross-spectrum coherence.

\section{Simulation Methods}
Numerical reference implementation uses a two-dimensional grid and an explicit scheme for the damped wave equation
\begin{equation}
  \partial_t^2 \Psi = c^2\,\nabla^2\Psi - k\,\partial_t\Psi + S\, .
\end{equation}
Convergence and stability are documented; an energy-like quantity is monitored (exact conservation in discrete time generally requires symplectic schemes, which we add in subsequent iterations). See repository path `simulations/` for code and configurations.

\section{List of Symbols}
\begin{itemize}
  \item $\Psi$ (A): scalar field amplitude.
  \item $\Phi_0$: substrate/background parameterization.
  \item $c$ ($L/T$): propagation speed.
  \item $k$ ($1/T$): damping coefficient in Rayleigh dissipation.
  \item $S$ ($A/T^2$): source term.
  \item $\hat M=\partial_t$, $\hat C=\nabla^2$, $\hat L=\nabla\cdot(\cdot)$.
\end{itemize}

\section{Discussion}
Limitations, falsifiable predictions, and future work.

\printbibliography

\end{document}
